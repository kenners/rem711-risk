%!TEX TS-program = xelatex
%!TEX encoding = UTF-8 Unicode

\documentclass[12pt,a4paper]{article}
\usepackage[margin=1.5in]{geometry}
\usepackage[parfill]{parskip}
\usepackage[superscript,biblabel]{cite}
\usepackage[UKenglish]{babel}
\usepackage[UKenglish]{isodate}
\usepackage{graphicx}

\usepackage[pdfauthor={Kenrick Turner}, pdftitle={A pre-deployment medical plan for Rothera Research Station, Antarctica}, pdfsubject={Antarctic medical risk assessment},pdfkeywords={antarctica, risk assessment, medical}]{hyperref}

% Need this to rotate table in appendix
\usepackage{pdflscape}
\usepackage{longtable}
\usepackage{booktabs}

\usepackage{fontspec,xltxtra,xunicode}
\defaultfontfeatures{Mapping=tex-text}
\setmainfont[Mapping=tex-text]{Helvetica Neue}
\setmonofont[Scale=MatchLowercase]{Source Code Pro}
\usepackage{microtype}

\title{A pre-deployment medical plan for Rothera Research Station, Antarctica}
\author{REM711: Pre-expedition Medical Plan}
\date{Student: 10463823 \\ Word Count: }

\linespread{1.3}

\begin{document}

\maketitle

% The student should address their own impending deployment to a remote environment and draw up a plan to provide optimal healthcare to the at risk population.  This does not need to address every hazard posed by the environment (although the range of hazards faced should be briefly described) but should cover at least three separate areas.

% In order to achieve this you will need to:

%   Describe the main areas of hazard to the expedition resulting from the environment, expedition activities or local populace. Consider also the specific interpersonal challenges of a lone practitioner caring for a group of which they are a part.  Choose areas that will be addressed with a rationale as to why they form the focus of the work.

%   Identify levels of risk associated with each hazard and describe techniques described in the literature to mitigate them.

%   Demonstrate an understanding of the harms likely to result from these hazards, and give an evidence-based strategy for managing them in the remote environment.

%   Identify the limitations of “standard” management of these conditions imposed by the constraints of the expedition (Transport, access to additional healthcare resources, portability, supply, cost etc).  Synthesize plans to provide optimal management within these limits.

%   Draw these plans together to provide kit lists, training objectives, procedures, strategies etc that demonstrate the optimal management of these hazards within your chosen environment.

%   Conclude with a summary of the key areas you need to address to keep your expedition safe from these hazards, highlighting how this may be different from conventional NHS practice.

\tableofcontents

\section{Introduction}

\begin{quote}
Proper Planning and Preparation Prevents Poor Performance.
\em British Army adage
\end{quote}

In October 2013, the British Antarctic Survey will begin its summer operations and resupply of Rothera Research Station, involving the transport of over a hundred personnel to Antarctica and a steady ramp up in scientific activities for the Austral summer. Personnel deploy to the continent for a variety of durations: from a mythical ``long-weekend in Rothera'' to an 18 month overwintering tour.

The Antarctic environment is frequently hostile and presents many hazards alien to the UK. Moreover, the ability to manage these hazards is severely constrained due to the limited infrastructure and personnel available. These limitations may require unorthodox or unusual solutions to be employed when devising a pragmatic medical plan.

Medical cover is provided by a single Medical Officer (MO), assisted by several Advanced First Aiders -- trained in part by the MO -- and all personnel have basic First Aid training prior to deployment. During the Austral summer, there is an overlap period of several months between incoming and outgoing MOs whereby two doctors may be available on site. Further medical support is rendered by a comprehensive telemedicine service by the British Antarctic Survey Medical Unit (BASMU), based in Derriford Hospital, Plymouth, UK. BASMU also trains MOs prior to deployment, which fosters a close working relationship between MOs and clinicians who will potentially be providing telemedicine support.

This medical plan documents the risk assessment that has been undertaken to identify the range of risks presented by Antarctic operations. Whilst addressing general considerations, hazards identified as taking the greatest importance will be discussed comprehensively, including strategies for avoidance and control of the encountered risks, and a response plan in the event of harm occurring.

\section{Situational Description}

As the largest British station on the continent, the summer population can exceed 120 people and the station serves as a busy airbase for

\section{Methodology}

\section{Challenges as the Lone Practitioner}

\section{Risk Assessment}

\section{Specific Hazards}

\subsection{Hazard 1}

\subsection{Hazard 2}

\subsection{Hazard 3}

\section{Conclusion}

\bibliographystyle{naturemag}
\bibliography{risk-assessment}

\begin{landscape}
    \appendix
    \section{Risk Assessment for Rothera Research Station}
    \begin{longtable}{lllll}
        \toprule
        \textbf{Activity} & \textbf{Hazard} & \textbf{Harm} & \textbf{Risk} & \textbf{Control Measures} \\
        \midrule
        \endhead
        Antarctic environment & Isolation & Depression/anxiety & Low & Prescreening \\
         &  &  &  & Encourage shared group activities \\
         &  &  &  & Regular communication with friends/family \\
         &  &  &  & Education and awareness \\
         & Confined environment & Interpersonal conflict & Moderate & Prescreening \\
         &  &  &  & Encourage shared group activities \\
         & Alcohol & Injury whilst intoxicated & Moderate & Education \\
         &  &  &  & Two can rule \\
         &  & Dependency/withdrawal & Low & Education \\
         &  &  &  & Prescreening \\
         &  &  &  & Encourage other recreational activities \\
         &  &  &  & Two can rule \\
         & Darkness & Seasonal affective disorder & Moderate & Prescreening \\
         &  &  &  & Encourage strict sleep hygeine \\
         &  &  &  & Light boxes \\
         &  &  &  & Vitamin D supplementation \\
        \midrule
        Field work & Cold & Hypothermia & Low & Education \\
         &  &  &  & Appropriate clothing \\
         &  &  &  & Adequate nutrition \\
         &  &  &  & Rewarming bath available on station \\
         &  & Frostbite/Frostnip & Moderate & Prescreening \\
         &  &  &  & Education \\
         &  &  &  & Appropriate clothing \\
         &  &  &  & Adequate hydration \\
         &  &  &  & Rewarming bath available on station \\
         &  & Non-freezing cold injury & Low & Education \\
         &  &  &  & Appropriate clothing \\
         &  &  &  & Adequate hydration \\
         &  & Dental sensitivity & Moderate & Prescreening \\
         &  &  &  & Provide high fluoride toothpaste \\
         &  &  &  & Duraphat varnish \\
         &  & Cracked tooth & Low & Prescreening \\
         &  &  &  & Education \\
         &  &  &  & Dental training for MO \\
         & Sun & Photokeratitis & Moderate & Education \\
         &  &  &  & Provision of Cat 4 wraparound sunglasses/goggles \\
         &  & Sunburn & Moderate & Education \\
         &  &  &  & Limit exposed skin \\
         &  &  &  & Provision of sunblock \\
         & Fire & Burns & Low & Education \\
         &  &  &  & Predeployment campcraft training \\
         & Cooking in tents & CO poisoning & Moderate & Education \\
         &  &  &  & Predeployment campcraft training \\
         &  &  &  & CO detectors in tents \\
         &  &  &  & Supplemental oxygen at bases \\
         & Low humidity & Dehydration & Moderate & Education \\
         &  &  &  & Encourage regular fluid intake \\
         & Contaminated water & Poisoning/dehydration & Low & Predeployment campcraft training \\
         &  &  &  & Use of Field Assistants \\
        \midrule
        Lab work & Hazardous chemicals & Exposure to harmful chemicals & Low & COSHH assessments to be completed \\
         &  &  &  & Use of appropriate PPE \\
         &  &  &  & BAS Lab induction training \\
         & Exposed flames & Burns & Moderate & BAS Lab induction training \\
         &  &  &  & Limit lone working where possible \\
         & Pressure vessels & Burns/scalds & Low & BAS Lab induction training \\
         &  &  &  & Limit lone working where possible \\
        \midrule
        Overland travel & Collision & Trauma & Low & Predeployment training \\
         &  &  &  & MAJAX plan and exercises \\
         & Getting lost & Various,  inc. trauma and cold injury & Moderate & Predeployment training \\
         &  &  &  & Use of Field Assistants \\
         &  &  &  & Provision of reserve fuel and emergency supplies \\
         & Avalanche & Major trauma/asphyxiation & Moderate & Travel routes to be approved by BAS Ops \\
         &  &  &  & Use of Field Assistants \\
         &  &  &  & Predeployment training \\
         &  &  &  & Meterology advice \\
         &  &  &  & MAJAX plan and exercises \\
         & Crevasse fall & Major trauma, cold injury & Moderate & Travel routes to be approved by BAS Ops \\
         &  &  &  & Use of Field Assistants \\
         &  &  &  & Predeployment training \\
         &  &  &  & MAJAX plan and exercises \\
        \midrule
        Aircraft travel & Noise & Hearing damage & Moderate & Use of appropriate PPE \\
         &  &  &  & Limit access to apron and runway during air ops \\
         & Altitude & Altitude illness (AMS, HACE, HAPE) & Low & Use minimum altitude circumstances permit \\
         &  &  &  & Prescreening \\
         &  &  &  & Prophylactic acetazolamide for susceptible individuals \\
         &  &  &  & Stagger direct flights to plateau \\
         & Propellers & Major trauma & Low & Limit access to apron and runway during air ops \\
         &  &  &  & Education \\
         &  &  &  & MAJAX plan and exercises \\
         & Crash & Major trauma & Low & Limit access to apron and runway during air ops \\
         &  &  &  & MAJAX plan and exercises \\
         & Manual handling & Crush, back, and hand injuries & Moderate & Predeployment training \\
         &  &  &  & Adequate manpower for task \\
         &  &  &  & Use of appropriate PPE \\
        \midrule
        Boat work & Capsize/Man overboard & Drowning & Low & Predeployment training \\
         &  &  &  & Regular MOB drills \\
         &  &  &  & MAJAX plan and exercises \\
         &  &  &  & Use of appropriate PPE \\
         &  & Hypothermia & Low & Use of appropriate PPE \\
         &  &  &  & Regular MOB drills \\
         &  &  &  & Predeployment training \\
         & Manual handling & Crush,  back, and hand injuries & Moderate & Predeployment training \\
         &  &  &  & Adequate manpower for task \\
         &  &  &  & Use of appropriate PPE \\
         & Engine failure/run aground & Various, inc hypothermia & Low & Boating ops to be approved by Boating officer \\
         &  &  &  & Appropriate clothing \\
         &  &  &  & Use of appropriate PPE \\
         &  &  &  & Reserve boat to be kept ready on station \\
        \midrule
        Diving & Cold water & Drowning & Low & Diving restricted to approved divers \\
         &  &  &  & Use of appropriate PPE \\
         &  &  &  & All diving to be approved by Diving officer \\
         &  &  &  & Absolute depth limit of 30m \\
         &  &  &  & No decompression diving \\
         &  &  &  & No solo diving \\
         &  & Hypothermia & Moderate & Use of appropriate PPE \\
         &  &  &  & No solo diving \\
         &  &  &  & Rewarming bath available on station \\
         &  & Decompression illness & Low & Diving restricted to approved divers \\
         &  &  &  & No nitrox or heliox dives \\
         &  &  &  & All diving to be approved by Diving officer \\
         &  &  &  & Absolute depth limit of 30m \\
         &  &  &  & No decompression diving \\
         &  &  &  & No solo diving \\
         &  &  &  & Decompression chamber on station \\
         &  & Ear infection & High & Education \\
         &  &  &  & Prompt treatment \\
         &  & Barotrauma & Moderate & Diving restricted to approved divers \\
         &  &  &  & Use of lines for descent and ascent \\
         &  &  &  & No solo diving \\
         & Fauna & Animal attack & Moderate & Diving restricted to approved divers \\
         &  &  &  & No solo diving \\
         &  &  &  & All diving to be approved by Diving officer \\
         &  &  &  & Lookouts to be posted for seals/orcas \\
         &  &  &  & No diving within 4hrs of sighting of seal/orca \\
         &  &  &  & Immediate termination of dive if seal/orca spotted \\
         &  &  &  & No decompression diving \\
        \midrule
        Outdoor recreation & Getting lost & Various,  inc trauma and cold injury & Moderate & Recreation restricted to approved areas \\
         &  &  &  & Predeployment training \\
         &  &  &  & Use of Field Assistants \\
         & Avalanche & Major trauma/asphyxiation & Moderate & Recreation restricted to approved areas \\
         &  &  &  & Use of Field Assistants \\
         &  &  &  & Predeployment training \\
         &  &  &  & Meterology advice \\
         &  &  &  & MAJAX plan and exercises \\
         & Crevasse fall & Major trauma, cold injury & Moderate & Recreation restricted to approved areas \\
         &  &  &  & Use of Field Assistants \\
         &  &  &  & Predeployment training \\
         &  &  &  & MAJAX plan and exercises \\
        \bottomrule
    \end{longtable}

\end{landscape}

\end{document}
