%!TEX TS-program = xelatex
%!TEX encoding = UTF-8 Unicode

\documentclass[12pt,a4paper]{article}
\usepackage[margin=1.5in]{geometry}
\usepackage[parfill]{parskip}
\usepackage[superscript,biblabel]{cite}
\usepackage[UKenglish]{babel}
\usepackage[UKenglish]{isodate}
\usepackage{graphicx}

\usepackage[unicode]{hyperref}
\hypersetup{pdfauthor={Kenrick Turner}, pdftitle={A pre-deployment medical plan for Rothera Research Station, Antarctica}, pdfsubject={Antarctic medical risk assessment},pdfkeywords={antarctica, risk assessment, medical}}

% Need this to rotate table in appendix
\usepackage{pdflscape}

% Pretty table stuff
\usepackage{tabu}
\usepackage{longtable}
\usepackage{booktabs}

\usepackage{fontspec,xltxtra,xunicode}
\defaultfontfeatures{Mapping=tex-text}
\setmainfont[Mapping=tex-text]{Helvetica Neue}
\setmonofont[Scale=0.9]{Source Code Pro}
\usepackage{microtype}

\title{A pre-deployment medical plan for Rothera Research Station, Antarctica}
\author{REM711: Pre-expedition Medical Plan}
\date{Student: 10463823 \\ Word Count: }

\linespread{1.3}

\begin{document}

\maketitle

\pagebreak

\tableofcontents

\pagebreak

\section{Introduction}

\begin{quote}
Proper Planning and Preparation Prevents Poor Performance. \\
\em British Army adage
\end{quote}

In October 2013, the British Antarctic Survey (BAS) will begin its summer operations and resupply of Rothera Research Station, involving the transport of over a hundred personnel to Antarctica and a steady ramp up in scientific activities for the Austral summer. Personnel deploy to the continent for a variety of durations: from a mythical ``long-weekend in Rothera'' to an 18 month overwintering tour.

The Antarctic environment is frequently hostile and presents many hazards alien to the UK. Moreover, the ability to manage these hazards is severely constrained due to the limited infrastructure and personnel available. These limitations may require unorthodox or unusual solutions to be employed when devising a pragmatic medical plan.

This medical plan documents the risk assessment that has been undertaken to identify the range of risks presented by Antarctic operations. Whilst addressing general considerations, hazards identified as taking the greatest importance will be discussed comprehensively, including strategies for avoidance and control of the encountered risks, and a response plan in the event of harm occurring.

\section{Situation Briefing}

\subsection{Environment}

Rothera research station was established in 1975 and is located at 67°34'S, 68°08'W on Adelaide Island on the Antarctic peninsula. It lies 1,900km south of the Falkland Isles and 1,500km south-east of Cape Horn. The terrain surrounding Rothera is mountainous with peaks in exceeding 2,500m and is also heavily glaciated. Temperatures range from around 0 -- 5ºC during summer to between -5ºC and -20ºC in winter, with gales and high winds being commonplace. Lying within the Antarctic circle, Rothera experiences several weeks without sunlight during the winter.

The station comprises of several low buildings, including accommodation facilities, workshops, laboratories, and a control tower. Air operations are facilitated by a hangar, fuel depot, and 900m long runway and apron in close proximity to accommodation buildings. As a coastal station, Rothera also has a wharf permitting resupply by ship.

Rothera is resupplied by both sea and air. Cargo is predominantly transported by sea on one of two calls per season by BAS-operated vessels, whilst personnel movements usually occur by air on Dash-7 or Twin Otter aircraft.

\subsection{Population}

As the largest research station operated by BAS, the summer population approaches 130 staff, whilst for the 2013-14 season approximately 20 people are expected to overwinter. Personnel ages range from early twenties through to early sixties. There is a significant gender skew with a 4:1 male:female ratio amongst overwintering staff. All personnel are prescreened prior to deployment which includes both medical and dental assessment.

\subsection{Activities}

 Scientific research consists mainly of marine biology, geology, glaciology and meteorology, with the majority of the work conducted during the Austral summer.

In addition to conducting research on and around station, Rothera also serves as a logistics hub for UK science parties and partner nations operating in the deep field, with personnel transiting Rothera either by air or by overland travel on snowmobile or snowcat. Deep field parties are geographically and temporally remote -- frequently hundreds kilometres and several hours flight time away.

In addition to local and deep field operations, two forward operating bases fall under Rothera's medical cover. Fossil Bluff (71°20'S, 68°17'W) and Sky-Blu (74°51'S, 71°34'W) are two logistics facilities around 90 minutes and 3 hours flight time from Rothera respectively. As summer-only stations, their role is to support deep field operations and have only limited personnel on site living in pyramid tents. Communication with Rothera is conducted via radio, and there are comprehensive medical kits at both sites.

\subsection{Medical Resources}

\subsubsection{Primary Medical Care}

Medical cover on station is provided by a single Medical Officer (MO), assisted by several Advanced First Aiders -- trained in part by the MO -- and all personnel have basic First Aid training prior to deployment. During the Austral summer, there is an overlap period of several months between incoming and outgoing MOs whereby two doctors may be available on site.

Further medical support is rendered by a comprehensive telemedicine service by the British Antarctic Survey Medical Unit (BASMU), based in Derriford Hospital, Plymouth, UK. BASMU also trains MOs prior to deployment, which fosters a close working relationship between MOs and clinicians who will potentially be providing telemedicine support.

The medical bay at Rothera is physically small but well-provisioned, including plain film radiology, full anaesthetic capability, and \emph{in extremis}, general surgical capability.

\subsubsection{Secondary Medical Care}

The closest secondary medical care facility is Clinica Magallanes\cite{Anonymous:TgM8hVWD} in Punta Arenas, Chile, and is 1,600km and 4.5 hours flight time away by Dash-7 aircraft. It is a private tertiary referral centre for the region, providing all major medical and surgical specialities, including neurosurgery. It has both CT and MRI imaging available onsite.

An alternative is Kind Edward VII Memorial Hospital in Stanley, Falkland Islands. It offers 17 acute beds, with an emergency department, dental services, a single theatre, and 2 ITU beds.\cite{Anonymous:qFBtPcmx} Whilst it can provide plain film radiology it has no CT scanner. It is a proven evacuation route, having previously received medivacs from Rothera, but it is further away than Punta (1,900km), incurring an additional hour's flight time (5.5 hours). Moreover, Punta offers a more comprehensive range of facilities so should be the preferred evacuation route.

Medical evacuation normally takes place by air, using Dash-7 aircraft which have a pressurised cabin. However, aircraft are not normally stationed at Rothera during winter, and BASMU maintains a policy of not undertaking evacuations during this period due to the risks presented by inclement weather. An alternative method is by ship, and casualties have previously been evacuated under exceptional circumstances during winter via this route. It should be stressed that evacuation may be delayed for a considerable period in both summer and winter owing to poor conditions.

\section{Risk Assessment}

Risk is an integral part of every challenging human endeavour. Indeed, it often contributes to the appeal of undertaking exigent activity itself. When devising a medical plan, the process of logically assessing likely hazards and their risk harm is a valuable planning tool. This plan employs a qualitative risk assessment model, and defines a hazard as any agent that may cause harm, whereas risk may be understood as a qualitative description of the composite of the severity of the harm and its likelihood in occurring.\cite{TheNationalPatientSafetyAgency:2007ud}

The Health and Safety Executive (HSE) describes five steps\cite{HSE:2012tp} in the process of risk assessment:

\begin{enumerate}
    \item Identify the hazards
    \item Decide who might be harmed and how
    \item Evaluate the risks and decide on precautions
    \item Record the findings and implement them
    \item Review the assessment and update as necessary
\end{enumerate}

Using this model, a brief qualitative risk assessment for Rothera station is undertaken in Appendix~\ref{sec:risk}.

When a risk is identified, four techniques may be employed to manage it:

\begin{description}
    \item[Avoidance] aims to reduce the \emph{probability} of risk i.e. reducing the likelihood of a harm occurring -- preferably to zero. Whilst effective, taken to an extreme it entails not performing the activity that exposes people to the hazard which is often impractical.
    \item[Mitigation] aims to reduce the \emph{severity} of a risk i.e. reducing the harm created when a risk occurs. It is useful in situations where risk is sizeable and cannot be easily avoided.
    \item[Transference] shifts or shares the harm of a risk when it occurs to a third party. Transferring risk is most appropriate for situations where the harm of a risk can be clearly measured and fully addressed e.g. duplicating mission-critical equipment in case of its loss.
    \item[Acceptance] is the strategy of actively deciding that one will accept the harm of a risk should it occur and planning appropriately to manage the consequences. Clearly the potential gains should heavily outweigh the harm, so acceptance is best reserved for when the severity of harm is significantly smaller than the cost of avoiding, mitigating, or transferring the risk.
\end{description}

Whilst a fully comprehensive risk assessment for Rothera is beyond the scope of this medical plan, three hazards have been identified that merit further discussion below.

\section{Challenges as the Lone Practitioner}

\begin{quote}
Physician, heal thyself. \\
\em Luke 4:23
\end{quote}

An inherent risk to Rothera is that for a significant proportion of the year there is a single medical provider. As a single point of failure, if the MO is unwell or injured, there is no other doctor to treat him. This risk may be mitigated by training advanced first aiders to a level where they can treat common conditions, and telemedicine support from BASMU transfers some of this risk. Furthermore, being aware of the risk, the MO may elect to avoid very high risk activities.

The doctor-patient relationship may also be altered when treating a colleague or friend, as professional judgement may be clouded by interpersonal issues. If there is a breakdown in friendship or working relationship, it may be difficult for an MO to act impartially. The facility to discuss issues with the BASMU telemedicine service provides a valuable sounding board for the MO in this context.

Moreover, the MO's actions may be scrutinised with disproportionate significance by other personnel, so the MO should be assiduously impartial to maintain their approachability and neutrality.

There is also a risk of developing a romantic relationship with patients in an isolated and confined environment, although this is discouraged. The GMC's guidance is unequivocal: ``You must not pursue a sexual or improper emotional relationship with a current patient.''\cite{Anonymous:2013wj,Anonymous:2013ve}

Confidentiality may also present a challenge, as the MO has a duty of care to not only the patient but to the station as a whole. There are therefore circumstances where the MO may need to breach confidentiality to the Base Commander to ensure the safety of others. Additionally, if a patient requires secondary medical care, logistical arrangements would also require limited disclosure of confidential information. Whenever the MO is disclosing confidential information, he should seek permission from the patient, explaining why and what is being disclosed.\cite{Anonymous:2013ta}

Tightly coupled to this issue is that the majority of the MO's work occurs in private, shielded from the rest of the station. Not only may this give rise to the impression of idleness by the MO, this situation also means the MO does not routinely have colleagues to vent or casually discuss confidences with -- often a key coping mechanism. Encouraging regular telephone contact with BASMU and other MOs on different stations is one mitigation strategy.

\section{Specific Hazards}

\subsection{Accidental hypothermia}

\subsubsection{Rationale}

Accidental hypothermia is a potentially life-threatening involuntary decrease of ≥2ºC in normal core body temperature. Whilst its incidence is reportedly low in Antarctica\cite{BritishAntarcticSurveyMedicalUnit:2013vj} (probably due to robust education and provision of appropriate clothing), mild cases may go unrecognised.\cite{Brown:2012ja} Further, a brief risk assessment (Appendix~\ref{sec:risk}) demonstrates that the cold environment is a hazard common to many regular activities undertaken at Rothera, with hypothermia being one of the major associated harms.

\subsubsection{Risk assessment}

The principle hazard for hypothermia is cold exposure. Hypothermia occurs when heat production is overcome by the stress of excessive cold. Consequently, there are many factors that increase the risk of hypothermia:

\begin{itemize}
    \item Environment
    \begin{itemize}
        \item Cold temperature
        \item Wind
        \item Wet clothing
        \item Altitude
    \end{itemize}
    \item Physiology
    \begin{itemize}
        \item Extremes of age
        \item Poor nutrition
        \item Dehydration
        \item Intoxication
        \item Medications
        \item Immobility
        \item Sleep deprivation
    \end{itemize}
    \item Comorbidities
        \begin{itemize}
            \item Trauma
            \item Hypothyroidism
        \end{itemize}
\end{itemize}

Many activities expose personnel to the cold at Rothera including field work, air travel, boat work, diving, and recreational activities such as skiing. Consequently, all personnel will be exposed to some risk, but the exact degree of risk will vary considerably. Particular high risk groups include divers and boatmen, deep field parties, aircrew and Field Assistants.

\subsubsection{Risk management}

The most effective strategy is to avoid exposing personnel to the cold. Prescreening of personnel prior to deployment will identify individuals at high risk who may not be suitable for Antarctic service. Other avoidance strategies include provision of adequate shelter/accommodation, provision of well-fitting appropriate clothing and personal protective equipment (PPE), adequate supply of hydration and nutrition, and other equipment specific to the activity (e.g. drysuits or warm-water suits for divers, and snowcats to be preferred over snowmobiles for overland travel in bad conditions).

It is unrealistic to expect personnel to avoid becoming cold at some point whilst deployed in Antarctica, hence mitigation strategies should also be employed to prevent coldness progressing to hypothermia. Hot drinks should be readily available, both on station and in the field, as should calorific snacks. Field parties should be equipped with means of establishing shelter quickly such as a bothy bag or pyramid tent. First aid kits should contain heat-reflective blankets and survival bags to facilitate rewarming. Boating and diving operations should plan to take spare warm clothing in case of becoming wet.

Coupled to both of these elements is education and training, as many of the steps outlined above rely on strict discipline and behaviour modification. Personnel are extensively briefed prior to deployment on fieldcraft to prevent hypothermia, how to recognise it in themselves and others, and how to initiate treatment.

\subsubsection{Management}

The diagnosis of hypothermia may be made either clinically or by measurement of core body temperature with a low-reading thermometer. The Swiss staging system for hypothermia provides the most pragmatic system for diagnosis and may also guide management (Table~\ref{tab:swiss}).\cite{Durrer:2003it} Severe hypothermia (HT IV) is notoriously difficult to distinguish from clinical death, giving rise to the adage ``nobody is dead until they are warm and dead.'' However, in the context of a non-permissive environment where other personnel may be at risk from exposure, a useful discriminator is an incompressible chest wall, indicating that the body is frozen solid and resuscitative efforts are futile.\cite{Brown:2012ja}

A variety of techniques are available for rewarming hypothermic patients:\cite{Brown:2012ja}

\begin{itemize}
    \item Passive rewarming
    \begin{itemize}
        \item Warm environment \& clothes
        \item Warm drinks
        \item Active movement
    \end{itemize}
    \item Active external rewarming
    \begin{itemize}
        \item Heat packs (charcoal/electric)
        \item Forced-air blankets (e.g. Barehugger)
    \end{itemize}
    \item Minimally invasive rewarming
    \begin{itemize}
        \item Warmed (40ºC) intravenous fluids
    \end{itemize}
    \item Invasive rewarming
    \begin{itemize}
        \item Without cardiac support
        \begin{itemize}
            \item Peritoneal dialysis
            \item Haemodialysis
            \item Thoracic lavage
        \end{itemize}
        \item With cardiac support
        \begin{itemize}
            \item Extracorporeal membrane oxygenation (ECMO)
            \item Cardiopulmonary bypass (CPB)
        \end{itemize}
    \end{itemize}
\end{itemize}

\begin{table}
    \begin{tabu} to\linewidth{X[l] X[2,l] X[c] X[2,l]}
        \toprule
        \rowfont\bfseries Stage & Symptoms & Core Temp. (ºC) & Treatment \\
        \midrule
        HT I & Conscious, shivering & 32--35 & Warm environment and clothing, warm sweet drinks, and active movement \\
        HT II & Impaired consciousness, no shivering & 28--32 & Cardiac monitoring, horizontal position \& immobilisation, active external rewarming (heat packs and blankets) \& minimally invasive rewarming (warm IV fluids) \\
        HT III & Unconscious, no shivering, vital signs present & 24--28 & As for HT II + airway management, consider rewarming with ECMO or cardio-pulmonary bypass \\
        HT IV & No vital signs, pliable chest wall & <24 & HT II \& III management + ALS, rewarming with ECMO or cardio-pulmonary bypass \\
        \bottomrule
    \end{tabu}
    \caption{The Swiss staging system for hypothermia\cite{Durrer:2003it}} \label{tab:swiss}
\end{table}

As shown in Table~\ref{tab:swiss}, treatment of hypothermia is guided by the Swiss stage. Nevertheless, management may be broadly divided into prehospital and hospital stages.

In the field, priorities should focus on provision of basic or advanced life support (BLS/ALS), passive and active rewarming, careful handling due to potential cardiac instability, and expedited transfer to the nearest medical facility.\cite{Brown:2012ja} CPR should only be commenced after careful checking for signs of life for at least one minute, and with the understanding that a prolonged resuscitation attempt may be required, potentially lasting hours.\cite{Soar:2010kd} Rescuers should be mindful of this requirement when undertaking such an attempt in a non- or semi-permissive environment. Rewarming in the prehospital setting should focus on heat packs and forced-air blankets, as other methods are either impractical or do not confer substantial heat transfer.\cite{Lundgren:2011ce,Hultzer:2005vn}

In the hospital setting, a hypothermic patient who is haemodynamically stable and normal should undergo active external and minimally invasive rewarming, as described above. This should produce a rise in core body temperature of \~2ºC/hr and be sufficient treatment for stage HT I and II hypothermia.\cite{Auerbach:2012tq,vanderPloeg:2010im} Patients presenting in stage HT III or IV hypothermia will often be haemodynamically compromised and may require cardiac support in the form of ECMO or CPB in addition to the rewarming techniques described for stages HT I and II.\cite{Brown:2012ja}

Hypothermic patients often have large intravenous fluid requirements due to cold diuresis and vasodilation associated with rewarming.\cite{Soar:2010kd} Balanced salt solutions should be used to avoid compounding a pre-existing metabolic acidosis.\cite{PowellTuck:2011us}

\subsubsection{Limitations of standard management}

Treatment of hypothermia in the field is limited to passive and active external rewarming, as invasive rewarming is clearly inappropriate and minimally invasive rewarming requires a prodigious amount of IV fluids. Furthermore, prehospital management may be complicated by difficulties in extricating and transporting a hypothermic patient, which in profound hypothermia (HT III and IV) may trigger cardiac arrest. On station, in addition to measures available in the field, minimally invasive rewarming may also be commenced. There is no capability to provide cardiac support, and over invasive rewarming techniques would be \emph{ad hoc} at best.

Therefore, hypothermic cardiac arrest in Antarctica is probably unsurvivable, owing to difficulties in transferring the patient to station, the lack of expertise and provision of ECMO or CPB, and the inability to provide Level 3 care. Moreover, attempts at resuscitation in the field are likely to place rescuers at risk themselves, despite their natural inclination to try and render first aid.

\subsubsection{Control Measures and Plan}

Given the limitations discussed above, the following control measures should be implemented.

\begin{enumerate}
    \item Predeployment screening \\
    Infirm individuals who are malnourished (e.g. annorexia nervosa) or who have other medical comorbidities increasing their risk of hypothermia should not be cleared for Antarctic service.
    \item Predeployment briefing \\
    The risk of hypothermia should be communicated to personnel and they should be briefed on prevention, recognition, and treatment of hypothermia. The effective ceiling of care should also be made clear.
    \item Hypothermic risk assessment for activities \\
    Comprehensive control measures for specific activities (e.g. diving, air travel etc.) is beyond the scope of this plan. Responsible individuals for these activities should conduct a risk assessment incorporating appropriate measures (e.g. Field Assistants should ensure appropriate shelter is available when planning field trips).
    \item Equipment for field medical kits \\
    Each field medical kit should contain the following items:
    \begin{itemize}
        \item 1 x  Blizzard Rescue Blanket
        \item 1 x North American Rescue Hypothermia Prevention and Management Kit
        \item 2 x TechTrade Ready-Heat 6 Panel Blanket
        \item 10 x 3M Tempadots Disposable Thermometers
        \item 1 x LESS Thermal Hood
        \item 1 x Lifesystems Bothy (4-6 man)
    \end{itemize}
    \item Equipment for Rothera station \\
    The medical bay at Rothera should be provisioned with the following items:
    \begin{itemize}
        \item 1 x  low-reading rectal thermometer
        \item 2 x North American Rescue Hypothermia Prevention and Management Kit
        \item 4 x TechTrade Ready-Heat 6 Panel Blanket
        \item 1 x LESS Thermal Hood
        \item 1 x 3M Bair Hugger 775 Temperature Management Unit
        \item 10 x 3M Bair Hugger 300 Full Body Blankets
        \item 1 x 3M Ranger Fluid Warming Unit
        \item 10 x 3M Ranger High Flow Disposable Fluid Warming Set (Model 24350)
    \end{itemize}
\end{enumerate}

\subsection{Carbon monoxide poisoning}

\subsubsection{Rationale}

\subsubsection{Risk assessment}

\subsubsection{Risk management}

\subsubsection{Management}

\subsubsection{Limitations of standard management}

\subsubsection{Control Measures and Plan}

\subsection{Seasonal Affective Disorder}

\subsubsection{Rationale}

\subsubsection{Risk assessment}

\subsubsection{Risk management}

\subsubsection{Management}

\subsubsection{Limitations of standard management}

\subsubsection{Control Measures and Plan}

\section{Conclusion}

\appendix
\begin{landscape}
    \section{Risk Assessment for Rothera Research Station}
    \label{sec:risk}
    \begin{longtabu} to\linewidth{X[l] X[l] X[l] X[l] X[2,l]}
        \toprule
        \rowfont\bfseries Activity & Hazard & Harm & Risk & Control Measures \\
        \midrule
        \endhead
        Antarctic environment & Isolation & Depression, anxiety & Low & Prescreening \\
         &  &  &  & Encourage shared group activities \\
         &  &  &  & Regular communication with friends and family \\
         &  &  &  & Education and awareness \\
         & Confined environment & Interpersonal conflict & Moderate & Prescreening \\
         &  &  &  & Encourage shared group activities \\
         & Alcohol & Injury whilst intoxicated & Moderate & Education \\
         &  &  &  & Two can rule \\
         &  & Dependency, withdrawal & Low & Education \\
         &  &  &  & Prescreening \\
         &  &  &  & Encourage other recreational activities \\
         &  &  &  & Two can rule \\
         & Darkness & Seasonal affective disorder & Moderate & Prescreening \\
         &  &  &  & Encourage strict sleep hygeine \\
         &  &  &  & Light boxes \\
         &  &  &  & Vitamin D supplementation \\
        \midrule
        Field work & Cold & Hypothermia & Moderate & Education \\
         &  &  &  & Appropriate clothing \\
         &  &  &  & Adequate nutrition \\
         &  & Frostbite, frostnip & Moderate & Prescreening \\
         &  &  &  & Education \\
         &  &  &  & Appropriate clothing \\
         &  &  &  & Adequate hydration \\
         &  &  &  & Rewarming bath available on station \\
         &  & Non-freezing cold injury & Low & Education \\
         &  &  &  & Appropriate clothing \\
         &  &  &  & Adequate hydration \\
         &  & Dental sensitivity & Moderate & Prescreening \\
         &  &  &  & Provide high fluoride toothpaste \\
         &  &  &  & Duraphat varnish \\
         &  & Cracked tooth & Low & Prescreening \\
         &  &  &  & Education \\
         &  &  &  & Dental training for MO \\
         & Sun & Photokeratitis & Moderate & Education \\
         &  &  &  & Provision of Cat 4 wraparound sunglasses \\
         &  & Sunburn & Moderate & Education \\
         &  &  &  & Limit exposed skin \\
         &  &  &  & Provision of sunblock \\
         & Fire & Burns & Low & Education \\
         &  &  &  & Predeployment campcraft training \\
         & Cooking in tents & CO poisoning & High & Education \\
         &  &  &  & Predeployment campcraft training \\
         &  &  &  & CO detectors in tents \\
         &  &  &  & Supplemental oxygen at bases \\
         & Low humidity & Dehydration & Moderate & Education \\
         &  &  &  & Encourage regular fluid intake \\
         & Contaminated water & Poisoning, dehydration & Low & Predeployment campcraft training \\
         &  &  &  & Use of Field Assistants \\
        \midrule
        Lab work & Hazardous chemicals & Exposure to harmful chemicals & Low & COSHH assessments to be completed \\
         &  &  &  & Use of appropriate PPE \\
         &  &  &  & BAS Lab induction training \\
         & Exposed flames & Burns & Moderate & BAS Lab induction training \\
         &  &  &  & Limit lone working where possible \\
         & Pressure vessels & Burns, scalds & Low & BAS Lab induction training \\
         &  &  &  & Limit lone working where possible \\
        \midrule
        Overland travel & Collision & Trauma & Low & Predeployment training \\
         &  &  &  & MAJAX plan and exercises \\
         & Getting lost & Various,  inc. trauma and cold injury & Moderate & Predeployment training \\
         &  &  &  & Use of Field Assistants \\
         &  &  &  & Provision of reserve fuel and emergency supplies \\
         & Avalanche & Major trauma, asphyxiation & Moderate & Travel routes to be approved by BAS Ops \\
         &  &  &  & Use of Field Assistants \\
         &  &  &  & Predeployment training \\
         &  &  &  & Meterology advice \\
         &  &  &  & MAJAX plan and exercises \\
         & Crevasse fall & Major trauma, cold injury & Moderate & Travel routes to be approved by BAS Ops \\
         &  &  &  & Use of Field Assistants \\
         &  &  &  & Predeployment training \\
         &  &  &  & MAJAX plan and exercises \\
        \midrule
        Aircraft travel & Noise & Hearing damage & Moderate & Use of appropriate PPE \\
         &  &  &  & Limit access to apron and runway during air ops \\
         & Altitude & Altitude illness (AMS, HACE, HAPE) & Low & Use minimum altitude circumstances permit \\
         &  &  &  & Prescreening \\
         &  &  &  & Prophylactic acetazolamide for susceptible individuals \\
         &  &  &  & Stagger direct flights to plateau \\
         & Propellers & Major trauma & Low & Limit access to apron and runway during air ops \\
         &  &  &  & Education \\
         &  &  &  & MAJAX plan and exercises \\
         & Crash & Major trauma & Moderate & Limit access to apron and runway during air ops \\
         &  &  &  & MAJAX plan and exercises \\
         & Manual handling & Crush, back, and hand injuries & Moderate & Predeployment training \\
         &  &  &  & Adequate manpower for task \\
         &  &  &  & Use of appropriate PPE \\
        \midrule
        Boat work & Capsize, man overboard & Drowning & Moderate & Predeployment training \\
         &  &  &  & Regular MOB drills \\
         &  &  &  & MAJAX plan and exercises \\
         &  &  &  & Use of appropriate PPE \\
         &  & Hypothermia & Low & Use of appropriate PPE \\
         &  &  &  & Regular MOB drills \\
         &  &  &  & Predeployment training \\
         & Manual handling & Crush,  back, and hand injuries & Moderate & Predeployment training \\
         &  &  &  & Adequate manpower for task \\
         &  &  &  & Use of appropriate PPE \\
         & Engine failure, run aground & Various, inc hypothermia & Low & Boating ops to be approved by Boating officer \\
         &  &  &  & Appropriate clothing \\
         &  &  &  & Use of appropriate PPE \\
         &  &  &  & Reserve boat to be kept ready on station \\
        \midrule
        Diving & Cold water & Drowning & Moderate & Diving restricted to approved divers \\
         &  &  &  & Use of appropriate PPE \\
         &  &  &  & All diving to be approved by Diving officer \\
         &  &  &  & Absolute depth limit of 30m \\
         &  &  &  & No decompression diving \\
         &  &  &  & No solo diving \\
         &  & Hypothermia & Moderate & Use of appropriate PPE \\
         &  &  &  & No solo diving \\
         &  &  &  & Rewarming bath available on station \\
         &  & Decompression illness & Low & Diving restricted to approved divers \\
         &  &  &  & No nitrox or heliox dives \\
         &  &  &  & All diving to be approved by Diving officer \\
         &  &  &  & Absolute depth limit of 30m \\
         &  &  &  & No decompression diving \\
         &  &  &  & No solo diving \\
         &  &  &  & Decompression chamber on station \\
         &  & Ear infection & High & Education \\
         &  &  &  & Prompt treatment \\
         &  & Barotrauma & Moderate & Diving restricted to approved divers \\
         &  &  &  & Use of lines for descent and ascent \\
         &  &  &  & No solo diving \\
         & Fauna & Animal attack & Moderate & Diving restricted to approved divers \\
         &  &  &  & No solo diving \\
         &  &  &  & All diving to be approved by Diving officer \\
         &  &  &  & Lookouts to be posted for seals/orcas \\
         &  &  &  & No diving within 4hrs of sighting of seal/orca \\
         &  &  &  & Immediate termination of dive if seal/orca spotted \\
         &  &  &  & No decompression diving \\
        \midrule
        Outdoor recreation & Getting lost & Various,  inc trauma and cold injury & Moderate & Recreation restricted to approved areas \\
         &  &  &  & Predeployment training \\
         &  &  &  & Use of Field Assistants \\
         & Avalanche & Major trauma, asphyxiation & Moderate & Recreation restricted to approved areas \\
         &  &  &  & Use of Field Assistants \\
         &  &  &  & Predeployment training \\
         &  &  &  & Meterology advice \\
         &  &  &  & MAJAX plan and exercises \\
         & Crevasse fall & Major trauma, cold injury & Moderate & Recreation restricted to approved areas \\
         &  &  &  & Use of Field Assistants \\
         &  &  &  & Predeployment training \\
         &  &  &  & MAJAX plan and exercises \\
        \bottomrule
    \end{longtabu}
\end{landscape}


\bibliographystyle{vancouver}
\bibliography{risk-assessment}

\end{document}
