%!TEX TS-program = xelatex
%!TEX encoding = UTF-8 Unicode

\documentclass[12pt,a4paper]{article}
\usepackage[margin=1.5in]{geometry}
\usepackage[parfill]{parskip}
\usepackage[superscript,biblabel]{cite}
\usepackage[UKenglish]{babel}
\usepackage[UKenglish]{isodate}
\usepackage{graphicx}

\usepackage[unicode]{hyperref}
\hypersetup{pdfauthor={Kenrick Turner}, pdftitle={A pre-deployment medical plan for Rothera Research Station, Antarctica}, pdfsubject={Antarctic medical risk assessment},pdfkeywords={antarctica, risk assessment, medical}}

% Need this to rotate table in appendix
\usepackage{pdflscape}
\usepackage{tabu}
\usepackage{longtable}
\usepackage{booktabs}

\usepackage{fontspec,xltxtra,xunicode}
\defaultfontfeatures{Mapping=tex-text}
\setmainfont[Mapping=tex-text]{Helvetica Neue}
\setmonofont[Scale=MatchLowercase]{Source Code Pro}
\usepackage{microtype}

\title{A pre-deployment medical plan for Rothera Research Station, Antarctica}
\author{REM711: Pre-expedition Medical Plan}
\date{Student: 10463823 \\ Word Count: }

\linespread{1.3}

\begin{document}

\maketitle

% The student should address their own impending deployment to a remote environment and draw up a plan to provide optimal healthcare to the at risk population.  This does not need to address every hazard posed by the environment (although the range of hazards faced should be briefly described) but should cover at least three separate areas.

% In order to achieve this you will need to:

%   Describe the main areas of hazard to the expedition resulting from the environment, expedition activities or local populace. Consider also the specific interpersonal challenges of a lone practitioner caring for a group of which they are a part.  Choose areas that will be addressed with a rationale as to why they form the focus of the work.

%   Identify levels of risk associated with each hazard and describe techniques described in the literature to mitigate them.

%   Demonstrate an understanding of the harms likely to result from these hazards, and give an evidence-based strategy for managing them in the remote environment.

%   Identify the limitations of “standard” management of these conditions imposed by the constraints of the expedition (Transport, access to additional healthcare resources, portability, supply, cost etc).  Synthesize plans to provide optimal management within these limits.

%   Draw these plans together to provide kit lists, training objectives, procedures, strategies etc that demonstrate the optimal management of these hazards within your chosen environment.

%   Conclude with a summary of the key areas you need to address to keep your expedition safe from these hazards, highlighting how this may be different from conventional NHS practice.

\tableofcontents

\section{Introduction}

\begin{quote}
Proper Planning and Preparation Prevents Poor Performance. \\
\em British Army adage
\end{quote}

In October 2013, the British Antarctic Survey (BAS) will begin its summer operations and resupply of Rothera Research Station, involving the transport of over a hundred personnel to Antarctica and a steady ramp up in scientific activities for the Austral summer. Personnel deploy to the continent for a variety of durations: from a mythical ``long-weekend in Rothera'' to an 18 month overwintering tour.

The Antarctic environment is frequently hostile and presents many hazards alien to the UK. Moreover, the ability to manage these hazards is severely constrained due to the limited infrastructure and personnel available. These limitations may require unorthodox or unusual solutions to be employed when devising a pragmatic medical plan.

This medical plan documents the risk assessment that has been undertaken to identify the range of risks presented by Antarctic operations. Whilst addressing general considerations, hazards identified as taking the greatest importance will be discussed comprehensively, including strategies for avoidance and control of the encountered risks, and a response plan in the event of harm occurring.

\section{Situation Briefing}

\subsection{Overview}

Rothera research station was established in 1975 and is located at 67°34'S, 68°08'W on Adelaide Island on the Antarctic peninsula. It lies 1,900km south of the Falkland Isles and 1,500km south-east of Cape Horn. The terrain surrounding Rothera is mountainous with peaks in exceeding 2,500m and is also heavily glaciated. Temperatures range from around 0 -- 5ºC during summer to between -5ºC and -20ºC in winter.

The station comprises of several low buildings, including accommodation facilities, workshops, laboratories, and a control tower. Air operations are facilitated by a hangar, fuel depot, and 900m long runway and apron in close proximity to accommodation buildings. As a coastal station, Rothera also has a wharf permitting resupply by ship.

Rothera is resupplied by both sea and air. Cargo is predominantly transported by sea on one of two calls per season by BAS-operated vessels, whilst personnel movements usually occur by air on Dash-7 or Twin Otter aircraft.

As the largest research station operated by BAS, the summer population approaches 130 staff, whilst for the 2013-14 season approximately 20 people are expected to overwinter. All personnel undergo health screening prior to deployment which includes both medical and dental assessment. Scientific research consists mainly of marine biology, geology, glaciology and meteorology, with the majority of the work conducted during the Austral summer.

In addition to conducting research on and around station, Rothera also serves as a logistics hub for UK science parties and partner nations operating in the deep field, with personnel transiting Rothera either by air or by overland travel on snowmobile or snowcat. Deep field parties are geographically and temporally remote -- frequently hundreds kilometres and several hours flight time away.

In addition to local and deep field operations, two forward operating bases fall under Rothera's medical cover. Fossil Bluff (71°20'S, 68°17'W) and Sky-Blu (74°51'S, 71°34'W) are two logistics facilities around 90 minutes and 3 hours flight time from Rothera respectively. As summer-only stations, their role is to support deep field operations and have only limited personnel on site living in pyramid tents. Communication with Rothera is conducted via radio, and there are comprehensive medical kits at both sites.

\subsection{Primary Medical Cover}

Medical cover on station is provided by a single Medical Officer (MO), assisted by several Advanced First Aiders -- trained in part by the MO -- and all personnel have basic First Aid training prior to deployment. During the Austral summer, there is an overlap period of several months between incoming and outgoing MOs whereby two doctors may be available on site.

Further medical support is rendered by a comprehensive telemedicine service by the British Antarctic Survey Medical Unit (BASMU), based in Derriford Hospital, Plymouth, UK. BASMU also trains MOs prior to deployment, which fosters a close working relationship between MOs and clinicians who will potentially be providing telemedicine support.

The medical bay at Rothera is physically small but well-provisioned, including plain film radiology, full anaesthetic capability, and \emph{in extremis}, general surgical capability.

\subsection{Secondary Medical Cover}

The closest secondary medical care facility is Clinica Magallanes\cite{Anonymous:TgM8hVWD} in Punta Arenas, Chile, and is 1,600km and 4.5 hours flight time away by Dash-7 aircraft. It is a private tertiary referral centre for the region, providing all major medical and surgical specialities, including neurosurgery. It has both CT and MRI imaging available onsite.

An alternative is Kind Edward VII Memorial Hospital in Stanley, Falkland Islands. It offers 17 acute beds, with an emergency department, dental services, a single theatre, and 2 ITU beds.\cite{Anonymous:qFBtPcmx} Whilst it can provide plain film radiology it has no CT scanner. It is a proven evacuation route, having previously received medivacs from Rothera, but it is further away than Punta (1,900km), incurring an additional hour's flight time (5.5 hours). Moreover, Punta offers a more comprehensive range of facilities so should be the preferred evacuation route.

Medical evacuation normally takes place by air, using Dash-7 aircraft which have a pressurised cabin. However, aircraft are not normally stationed at Rothera during winter, and BASMU maintains a policy of not undertaking evacuations during this period due to the risks presented by inclement weather. An alternative method is by ship, and casualties have previously been evacuated under exceptional circumstances during winter via this route. It should be stressed that evacuation may be delayed for a considerable period in both summer and winter owing to poor conditions.

\section{Risk Assessment}

Risk is an integral part of every challenging human endeavour. Indeed, it often contributes to the appeal of undertaking exigent activity itself. When devising a medical plan, the process of logically assessing likely hazards and their risk harm is a valuable planning tool. For the purposes of this document, a hazard is any agent that may cause harm, whereas risk may be understood as the composite of the severity of the harm and its likelihood.

In qualitative risk assessments, risk is defined as a set of single values combining both the severity and likelihood of harm (e.g. ``Low'', ``Moderate'', and ``High'').\cite{TheNationalPatientSafetyAgency:2007ud} A more detailed description of risk may be produced through \emph{quantitative} or probabilistic risk assessment (PRA), whereby risk is described by individually scoring severity and probability of harm -- usually both on an arbitrary scale of 1-5 -- with the product giving a Composite Risk Index.\cite{Stamatelatos:2012vn,NationalReportingandLearningService:2007ts} This provides a framework to compare severe but rare harm to be compared to common, trivial harm.

The Health and Safety Executive (HSE) describes five steps\cite{HSE:2012tp} in the process of risk assessment:

\begin{enumerate}
    \item Identify the hazards
    \item Decide who might be harmed and how
    \item Evaluate the risks and decide on precautions
    \item Record the findings and implement them
    \item Review the assessment and update as necessary
\end{enumerate}

Using this model, a brief qualitative risk assessment for Rothera station is presented in Appendix A. Whilst a fully comprehensive risk assessment is beyond the scope of this medical plan, three hazards have been identified that merit further discussion below.

\section{Challenges as the Lone Practitioner}

\begin{quote}
Physician, heal thyself. \\
\em Luke 4:23
\end{quote}

\section{Specific Hazards}

\subsection{Hazard 1}

\subsection{Hazard 2}

\subsection{Hazard 3}

\section{Conclusion}

\bibliographystyle{naturemag}
\bibliography{risk-assessment}

\appendix
\begin{landscape}
    \section{Risk Assessment for Rothera Research Station}
    \begin{longtabu} to\linewidth{X[l]X[l]X[l]X[l]X[2,l]}
        \toprule
        \rowfont\bfseries Activity & Hazard & Harm & Risk & Control Measures \\
        \midrule
        \endhead
        Antarctic environment & Isolation & Depression, anxiety & Low & Prescreening \\
         &  &  &  & Encourage shared group activities \\
         &  &  &  & Regular communication with friends and family \\
         &  &  &  & Education and awareness \\
         & Confined environment & Interpersonal conflict & Moderate & Prescreening \\
         &  &  &  & Encourage shared group activities \\
         & Alcohol & Injury whilst intoxicated & Moderate & Education \\
         &  &  &  & Two can rule \\
         &  & Dependency, withdrawal & Low & Education \\
         &  &  &  & Prescreening \\
         &  &  &  & Encourage other recreational activities \\
         &  &  &  & Two can rule \\
         & Darkness & Seasonal affective disorder & Moderate & Prescreening \\
         &  &  &  & Encourage strict sleep hygeine \\
         &  &  &  & Light boxes \\
         &  &  &  & Vitamin D supplementation \\
        \midrule
        Field work & Cold & Hypothermia & Low & Education \\
         &  &  &  & Appropriate clothing \\
         &  &  &  & Adequate nutrition \\
         &  &  &  & Rewarming bath available on station \\
         &  & Frostbite, frostnip & Moderate & Prescreening \\
         &  &  &  & Education \\
         &  &  &  & Appropriate clothing \\
         &  &  &  & Adequate hydration \\
         &  &  &  & Rewarming bath available on station \\
         &  & Non-freezing cold injury & Low & Education \\
         &  &  &  & Appropriate clothing \\
         &  &  &  & Adequate hydration \\
         &  & Dental sensitivity & Moderate & Prescreening \\
         &  &  &  & Provide high fluoride toothpaste \\
         &  &  &  & Duraphat varnish \\
         &  & Cracked tooth & Low & Prescreening \\
         &  &  &  & Education \\
         &  &  &  & Dental training for MO \\
         & Sun & Photokeratitis & Moderate & Education \\
         &  &  &  & Provision of Cat 4 wraparound sunglasses \\
         &  & Sunburn & Moderate & Education \\
         &  &  &  & Limit exposed skin \\
         &  &  &  & Provision of sunblock \\
         & Fire & Burns & Low & Education \\
         &  &  &  & Predeployment campcraft training \\
         & Cooking in tents & CO poisoning & Moderate & Education \\
         &  &  &  & Predeployment campcraft training \\
         &  &  &  & CO detectors in tents \\
         &  &  &  & Supplemental oxygen at bases \\
         & Low humidity & Dehydration & Moderate & Education \\
         &  &  &  & Encourage regular fluid intake \\
         & Contaminated water & Poisoning, dehydration & Low & Predeployment campcraft training \\
         &  &  &  & Use of Field Assistants \\
        \midrule
        Lab work & Hazardous chemicals & Exposure to harmful chemicals & Low & COSHH assessments to be completed \\
         &  &  &  & Use of appropriate PPE \\
         &  &  &  & BAS Lab induction training \\
         & Exposed flames & Burns & Moderate & BAS Lab induction training \\
         &  &  &  & Limit lone working where possible \\
         & Pressure vessels & Burns, scalds & Low & BAS Lab induction training \\
         &  &  &  & Limit lone working where possible \\
        \midrule
        Overland travel & Collision & Trauma & Low & Predeployment training \\
         &  &  &  & MAJAX plan and exercises \\
         & Getting lost & Various,  inc. trauma and cold injury & Moderate & Predeployment training \\
         &  &  &  & Use of Field Assistants \\
         &  &  &  & Provision of reserve fuel and emergency supplies \\
         & Avalanche & Major trauma, asphyxiation & Moderate & Travel routes to be approved by BAS Ops \\
         &  &  &  & Use of Field Assistants \\
         &  &  &  & Predeployment training \\
         &  &  &  & Meterology advice \\
         &  &  &  & MAJAX plan and exercises \\
         & Crevasse fall & Major trauma, cold injury & Moderate & Travel routes to be approved by BAS Ops \\
         &  &  &  & Use of Field Assistants \\
         &  &  &  & Predeployment training \\
         &  &  &  & MAJAX plan and exercises \\
        \midrule
        Aircraft travel & Noise & Hearing damage & Moderate & Use of appropriate PPE \\
         &  &  &  & Limit access to apron and runway during air ops \\
         & Altitude & Altitude illness (AMS, HACE, HAPE) & Low & Use minimum altitude circumstances permit \\
         &  &  &  & Prescreening \\
         &  &  &  & Prophylactic acetazolamide for susceptible individuals \\
         &  &  &  & Stagger direct flights to plateau \\
         & Propellers & Major trauma & Low & Limit access to apron and runway during air ops \\
         &  &  &  & Education \\
         &  &  &  & MAJAX plan and exercises \\
         & Crash & Major trauma & Low & Limit access to apron and runway during air ops \\
         &  &  &  & MAJAX plan and exercises \\
         & Manual handling & Crush, back, and hand injuries & Moderate & Predeployment training \\
         &  &  &  & Adequate manpower for task \\
         &  &  &  & Use of appropriate PPE \\
        \midrule
        Boat work & Capsize, man overboard & Drowning & Low & Predeployment training \\
         &  &  &  & Regular MOB drills \\
         &  &  &  & MAJAX plan and exercises \\
         &  &  &  & Use of appropriate PPE \\
         &  & Hypothermia & Low & Use of appropriate PPE \\
         &  &  &  & Regular MOB drills \\
         &  &  &  & Predeployment training \\
         & Manual handling & Crush,  back, and hand injuries & Moderate & Predeployment training \\
         &  &  &  & Adequate manpower for task \\
         &  &  &  & Use of appropriate PPE \\
         & Engine failure, run aground & Various, inc hypothermia & Low & Boating ops to be approved by Boating officer \\
         &  &  &  & Appropriate clothing \\
         &  &  &  & Use of appropriate PPE \\
         &  &  &  & Reserve boat to be kept ready on station \\
        \midrule
        Diving & Cold water & Drowning & Low & Diving restricted to approved divers \\
         &  &  &  & Use of appropriate PPE \\
         &  &  &  & All diving to be approved by Diving officer \\
         &  &  &  & Absolute depth limit of 30m \\
         &  &  &  & No decompression diving \\
         &  &  &  & No solo diving \\
         &  & Hypothermia & Moderate & Use of appropriate PPE \\
         &  &  &  & No solo diving \\
         &  &  &  & Rewarming bath available on station \\
         &  & Decompression illness & Low & Diving restricted to approved divers \\
         &  &  &  & No nitrox or heliox dives \\
         &  &  &  & All diving to be approved by Diving officer \\
         &  &  &  & Absolute depth limit of 30m \\
         &  &  &  & No decompression diving \\
         &  &  &  & No solo diving \\
         &  &  &  & Decompression chamber on station \\
         &  & Ear infection & High & Education \\
         &  &  &  & Prompt treatment \\
         &  & Barotrauma & Moderate & Diving restricted to approved divers \\
         &  &  &  & Use of lines for descent and ascent \\
         &  &  &  & No solo diving \\
         & Fauna & Animal attack & Moderate & Diving restricted to approved divers \\
         &  &  &  & No solo diving \\
         &  &  &  & All diving to be approved by Diving officer \\
         &  &  &  & Lookouts to be posted for seals/orcas \\
         &  &  &  & No diving within 4hrs of sighting of seal/orca \\
         &  &  &  & Immediate termination of dive if seal/orca spotted \\
         &  &  &  & No decompression diving \\
        \midrule
        Outdoor recreation & Getting lost & Various,  inc trauma and cold injury & Moderate & Recreation restricted to approved areas \\
         &  &  &  & Predeployment training \\
         &  &  &  & Use of Field Assistants \\
         & Avalanche & Major trauma, asphyxiation & Moderate & Recreation restricted to approved areas \\
         &  &  &  & Use of Field Assistants \\
         &  &  &  & Predeployment training \\
         &  &  &  & Meterology advice \\
         &  &  &  & MAJAX plan and exercises \\
         & Crevasse fall & Major trauma, cold injury & Moderate & Recreation restricted to approved areas \\
         &  &  &  & Use of Field Assistants \\
         &  &  &  & Predeployment training \\
         &  &  &  & MAJAX plan and exercises \\
        \bottomrule
    \end{longtabu}
\end{landscape}

\end{document}
